%%%%%%
% Beginning of file sample.tex 
%
% Acta Spartae Sample File
% Ethan Deneault September 2016
%
%%%%%%

% The following commands load the class file and the hyperlink reference package. 

\documentclass[final]{actaspartae}
\usepackage{hyperref}

% Do not edit this. This is for publication information

\begin{document}
\pubmonth{September}
\pubyear{2018}
\volume{4}
\issue{1}
\setcounter{page}{3}
\firstpage{\thestartpage}
\lastpage{\getpagerefnumber{bottom}}


% License codes are for the author to license their work via one of the creative commons licenses or by a generic license. 
%
% acta - Generic license
% by - CC License Attribution
% by-sa - CC License Attribution-Share-Alike
% by-nc-sa - CC License Attribution-Noncommercial-ShareAlike
% zero - Public Domain Dedication 

\license{acta}

% Comment out \raggedend if you want to balance out the columns on the last page. 

%\raggedend

% Title and subject information goes here

\subjectarea{Science} 

\title[This is a short title]{This is the full title of the article}

\author[1]{Bob Student}
\author[2]{Jane Student}
\author[2,3]{Faculty Person}

\affil[1]{Department of Something} 
\affil[2]{Department of Something Else, University of Tampa, Tampa, FL 33606} 
\affil[3]{Faculty Advisor}

% This makes the title block. 

\maketitle

% This is your abstract. 

\begin{abstract}
This abstract is super exciting! I did some science and it was fun. I got some results. If you read my paper, It'll tell you all about it. 
\end{abstract}

% Use the \section{} command to start a section. If there are subsections, use \subsection{}. 
% Use \citet{} and \citep{} for citations in text and in parentheses, respectively. 

\section{Introduction}

Science. It is what happens in a laboratory. The best kind of science references papers that other people did. \citet{carrier70} say that we should look at bacteria, because they are cool, but I think they aren't as cool as toxicology \citep{downs16}. 

\section{Materials and Methods}

I used some stuff. Science stuff. 

\subsection{More information about x, y and z}

A wild table appears: 

\begin{table}[th!]
\centering
\fontsize{8}{10}\selectfont
\begin{tabular}{l >{\centering\arraybackslash}p{7em}}
\multicolumn{1}{c}{\textbf{Species}} & \multicolumn{1}{c}{\textbf{MLD}} \\
\toprule
\emph{Escherichia coli} & \SI{40}{\second}\\
\emph{Micrococcus luteus} & \SI{120}{\second}\\
\emph{Micrococcus radiophilus} & \SI{240}{\second}\\
\botrule
\end{tabular}
\caption{Determined MLD of E. coli, M. luteus, and M. radiophilus. The Micrococcus species   have a substantially higher MLD when compared to E. coli. Signifies the sensitivity of E. coli when exposed alone to UV radiation. }
\label{tab:bugs}  %% Label your tables for referencing your tables!
\end{table}
See, you can reference table \ref{tab:bugs} easily using \verb+\label{}+. 


\section{Results}

I got results. 

\section{Discussion/Conclusion}

Revel in my scientific prowess. 

\acknowledgements

I'd like to thank the academy and NSF for all the grant monies. 

% References must be in the format shown below. Note spaces (or lack of spaces) carefully! 
% \bibitem[authornames(year)]lastname, A. B. year \href{DOI number}{Journal abbreviated Name and Pages}
% Journal abbreviations can be found here: https://library.caltech.edu/reference/abbreviations/

\begin{thebibliography}{99}

\bibitem[Carrier \& Setlow(1970)]{carrier70}Carrier, W. L. \& Setlow, R. B. 1970. \href{http://www.ncbi.nlm.nih.gov/pmc/articles/PMC284985/}{J Bacteriol, 102:178--186}

\bibitem[Downs et al.(2016)]{downs16}Downs, C.A., Kramarsky-Winter, E., Segal, R. et al. \href{http://dx.doi.org/10.1007/s00244-015-0227-7}{Arch Environ Contam Toxicol 70: 265}

\end{thebibliography}



\label{bottom}
\end{document}